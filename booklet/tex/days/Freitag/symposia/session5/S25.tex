
            \begin{symposium}
            {S25 - Mnemonic processing of immersive environments: Neuronal findings on different memory systems from virtual reality studies }
            {Session chair(s):  Joanna Kisker }
            {Freitag  - 12:30 | Ort TODO}
            {Welche Uni TODO}
            Autor Joanna Kisker 

The majority of everyday memories is based on sensory-rich, three-dimensional experiences. For that reason, Virtual Reality (VR) is increasingly used to approximate realistic experiences. Yet little is known about how the neuronal correlates of memory derived from 2D-conditions translate to immersive conditions. Consequently, the symposium explores how different memory systems operate under VR-conditions, demonstrating both fundamental principles and practical applications. 
Anna Vorreuther presents a series of VR-fNIRS-studies examining the neuronal correlates of working memory load and associated learning progress. She demonstrates how brain-computer-interfaces can be utilized to develop and tailor immersive learning systems to individual needs and abilities. 
As Felix Klotzsche demonstrates, visual short-term memory is affected by the spatial relationship between stimulus and observer. By assessing the spatial constraints underlying two well-established electrophysiological markers of memory retention, he examines the effects of stimulus eccentricity. 
Likewise, spatial memory is facilitated by offering real-time 3D-content: Julia Belger presents an immersive Virtual Memory Task which allows for assessing, training and rehabilitating spatial memory deficits in neurologic patients, demonstrating the advantages of using VR in neuropsychological practice.
To unravel the dependence of episodic memory retrieval on the encoding modality, Joanna Kisker compares the electrophysiological correlates of retrieval of VR-based and 2D-based engrams, and demonstrates the potential to refine these findings by examining the high-frequency domain.
Concluding, Marike Johnsdorf presents a comprehensive investigation on how different degrees of reality affect object perception and mnemonic processing. Remarkably, she contrasts the neuronal correlates of a conventional laboratory, a realistic VR, and a real-life condition.
            \begin{description}    
            
                \item [Anna Vorreuther  (TODO: refactor name)] TODO: add title \textcolor{mygray}{ | 11:00}    
                
                \item [Felix Klotzsche  (TODO: refactor name)] TODO: add title \textcolor{mygray}{ | 11:15}    
                
                \item [J. Belger (TODO: refactor name)] TODO: add title \textcolor{mygray}{ | 11:30}    
                
                \item [Joanna Kisker (TODO: refactor name)] TODO: add title \textcolor{mygray}{ | 11:45}    
                
                \item [Marike Johnsdorf (TODO: refactor name)] TODO: add title \textcolor{mygray}{ | 12:00}    
                
            \end{description} 
            \end{symposium}
            