
            \begin{symposium}
            {S23 - The Brain on Gonadal Hormones: Uncovering the Interplay between Affect and Brain Dynamics}
            {Anna Denninger}
            {Freitag 12:45 - 14:15 | Room 1.18}
            {Biological and Clinical Psychology, University of Trier and Institute for Cognitive and Affective Neuroscience}
            Gonadal hormones, integral to the reproductive system in both sexes, appear to play a critical role in our everyday life regulating various physiological and psychological responses within the body and influencing our affective system including the stress response, emotion regulation, reward processing and mood. As the brain represents a gateway for endocrine effects, their influence further extends to brain structure and functioning. Any alterations in the hormone levels may lead to health challenges. Women, especially, experience hormonal fluctuation throughout their lifespan that impact brain function and plasticity, affect regulation, and mental well-being. Thus, changes in hormonal status (e.g., menstrual cycle fluctuations, use of hormonal contraceptives, and menopause) have been linked to mental health and brain architecture changes in different groups of women. Our symposium focuses on current state-of-the-art research on the interplay between gonadal hormones, affect and brain functioning. Gregor Domes (Trier) discusses a meta-analysis on stress reactivity’s link to gonadal hormones. Followed up by Anna Denninger (Tübingen) exploring the impact of experimentally elevated estrogen on brain volume and emotion regulation in women. Tobias Sommer (Hamburg) then presents data on brain function, reward processing, and reinforcement learning in both sexes. Next, Ann-Christin Kimmig (Tübingen) analyzes inter-subject representational similarity in women discontinuing or starting oral contraception, examining hormone concentration variability, resting-state functional connectivity, and depression. Arielle Crestol (Oslo) investigates the link between proxies of cellular and brain aging with menopause-related factors, depression, and APOE $\epsilon$4 genotype. Overall, this symposium will explore the complex interplay between gonadal hormones, affect, and brain dynamics.
            \begin{description}    
            
                \item [ Domes G.] The acute effects of psychosocial stress on gonadal steroid secretion in humans – a meta analysis \textcolor{mygray}{ | 12:45}    
                
                \item [ Denninger A.] Effects of estradiol and emotion regulation on grey matter volume  \textcolor{mygray}{ | 13:00}    
                
                \item [ Sommer T.] Influence of estrogen on dopamine-related brain activity \textcolor{mygray}{ | 13:15}    
                
                \item [ A-C.S. K.] Navigating Mood: Understanding Oral Contraceptives' Influence on Mental Well-Being  \textcolor{mygray}{ | 13:30}    
                
                \item [ Crestol A.] Proxies of biological aging are associated with menopause, depression, and genetic risk for Alzheimer’s disease in females \textcolor{mygray}{ | 13:45}    
                
            \end{description} 
            \end{symposium}
            