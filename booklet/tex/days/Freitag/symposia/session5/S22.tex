
            \begin{symposium}
            {S22 - Uncovering lifespan signatures: A multimodal perspective at the interface of development, aging, and disease risk }
            {Christina Artemenko}
            {Freitag 12:45 - 14:15 | Large auditorium}
            {TU Dresden}
            In our modern aging society, individuals must function in their daily life well into old age. Cognitive deficits during aging might therefore have a detrimental impact on the ability to live independently. Hence, it is essential to better understand how cognitive processes change during aging.
This symposium addresses this question in the domains of perception, memory, and numerical cognition. Thus, age-related effects will be discussed regarding multisensory plausibility, visual distractibility, episodic and working memory, as well as number processing and arithmetic. The employed tasks cover the whole range from basic to complex cognitive performance tests.
In addition to behavioral methods, functional (fNIRS) and structural (MRI) neuroimaging techniques were used to identify the underlying neural mechanisms subserving cognitive functions. Methodologically, study designs consisted of cross-sectional studies (comparison of older and younger adults), longitudinal studies (developmental changes during aging), intervention studies (pre-post-design with a control group), and patient studies (neurogenerative disease with or without cognitive impairment and a healthy control group). This methodological variety reflects the chances and challenges in the research field on cognitive aging.
The findings reveal age-related deficits in subjective perception and objective performance, but also age-related modulation and compensation mechanisms that support the preservation of cognitive functions during aging.
            \begin{description}    
            
                \item [ Li S.] Aging and digitalized perceptual augmentation: Lessons learned from cortical processes of multisensory plausibility in virtual environments \textcolor{mygray}{ | 12:45}    
                
                \item [ Klink H.] The degree of subjective cognitive complaints is related to increased distractibility but also increased improvement in visual processing speed after physical exercise \textcolor{mygray}{ | 13:00}    
                
                \item [ Dahl M.] The integrity of dopaminergic and noradrenergic brain regions is associated with different aspects of late-life memory performance \textcolor{mygray}{ | 13:15}    
                
                \item [ Artemenko C.] Age-related changes in arithmetic in the fronto-parietal network \textcolor{mygray}{ | 13:30}    
                
                \item [ Loenneker H.] Basic numerical cognition, arithmetic, and activities of daily living in Parkinson’s Disease \textcolor{mygray}{ | 13:45}    
                
            \end{description} 
            \end{symposium}
            