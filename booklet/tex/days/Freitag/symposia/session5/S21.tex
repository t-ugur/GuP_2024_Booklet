
            \begin{symposium}
            {S21 - Unravelling Visual Prediction: Insights from Electrophysiology, EEG, fMRI, and Computational Modelling}
            {Helen Blank}
            {Freitag  - 12:30 | Ort TODO}
            {Neural Circuits and Cognition Lab, European Neuroscience Institute Göttingen Perception and Plasticity Group, German Primate Center}
            The central role of predictions in visual perception prompts ongoing debates regarding how priors influence sensory processing, specifically whether, where, and when they increase or reduce representations of expected input. This symposium unifies researchers employing diverse methodologies, encompassing electrophysiology in the primate brain as well as fMRI, EEG, and computational modelling based on deep convolutional neural networks (DCNN) in humans, with the collective aim to understand how predictions shape visual processing.
In the domain of hierarchical face recognition, Caspar Schwiedrzik will show that tuning properties in early regions of the macaque face-processing system reflect properties of higher areas, revealing the flexible transformation of representational spaces by predictive context. Correspondingly, Annika Garlichs employs multivariate fMRI analyses with DCNNs to demonstrate prediction-dependent error processing throughout, as well as heightened representations at early stages of the face-processing hierarchy in humans. In the domain of image processing, Lea-Maria Schmitt presents a series of behavioural and fMRI studies with laminar precision probing the recurrent dynamics underlying the perception of novel but not familiar images. Arjen Alink presents EEG evidence that initially predictions facilitate processing of expected visual information in natural images while later enhancing the processing of unexpected input, thereby suggesting that priors and input are differentially integrated over time. Finally, Wanlu Fu combines a DNN model with EEG recordings to show that readers optimize visual information using predictive coding principles by focusing on the orthographic prediction error. Overall, the symposium will provide computational insights into how predictions influence neural representations across visual processing hierarchies.
            \begin{description}    
            
                \item [ Schwiedrzik C.] Linking pattern separation to predictive processing in high-level visual cortex  \textcolor{mygray}{ | 11:00}    
                
                \item [ Garlichs A.] Computational Modelling Reveals Prediction Error and Sharpened Representations Across the Face-Processing Hierarchy \textcolor{mygray}{ | 11:15}    
                
                \item [ Alink A.] Stimulus-evoked EEG response patterns more strongly encode expected and unexpected image components consecutively \textcolor{mygray}{ | 11:30}    
                
                \item [ Schmitt L.] What recurrent dynamics underlie the perception of familiar and novel images? \textcolor{mygray}{ | 11:45}    
                
                \item [ Fu W.] Specifying the orthographic prediction error for a better understanding of efficient visual word recognition in humans and machines \textcolor{mygray}{ | 12:00}    
                
            \end{description} 
            \end{symposium}
            