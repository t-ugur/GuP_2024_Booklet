
            \begin{symposium}
            {S27 - Improving replicability in neuroscientific research (IGOR-Symposium)}
            {Hilmar Zech}
            {Freitag 14:30 - 16:00 | Large auditorium}
            {University of Bielefeld, University Medical Center Hamburg-Eppendorf}
            The topic of replicability has been widely discussed in neuroscientific research recently. Replicability challenges have been particularly highlighted for experimental tasks, which are crucial for bridging the gap between brain and behavior. Therefore, this symposium will focus on improving replicability in neuroscientific research, with a special emphasis on testing and enhancing the reliability of experimental tasks. Tina Lonsdorf will start off the symposium by providing a general introduction to the topic of replicability with a focus on neuroscientific research. She will introduce the multiverse approach as a potential solution to improve replicability and share a vision for a living database. Next, Sercan Kahveci will offer a detailed comparison of methods for determining the split-half reliability of experimental tasks while highlighting best practices and giving hands-on recommendations for neuroscience researchers. Building upon this talk, Hilmar Zech will demonstrate how test-retest reliability of smartphone-based experimental tasks can be improved by pooling longitudinal data, and how this can improve neuroscientific research by linking task outcomes to real-world behaviors. Advancing to fMRI research, Juliane Nagel will showcase, how large-scale online experiments can be instrumental in improving the reliability and, ultimately, the replicability of costly fMRI experiments. Finally, Nils Kroemer will underscore the importance of assessing individual-level reliability in task research to foster translational research that links task outcomes to neurological disorders. Together, this symposium will highlight the importance of replicability and provide researchers with insights into the toolkit necessary to promote reliable and replicable research in neuroscience.
            \begin{description}    
            
                \item [ Lonsdorf T.] Navigating replicability in experimental behavioral neuroscience \textcolor{mygray}{ | 14:30}    
                
                \item [ Kahveci S.] Reliability of reaction time tasks: exploring the methods and pitfalls of its computation \textcolor{mygray}{ | 14:45}    
                
                \item [ Zech H.] Improving task reliability in experimental behavioral neuroscience \textcolor{mygray}{ | 15:00}    
                
                \item [ Nagel J.] Using online-studies to perform precise human neuroscience: how do rewards affect long-term memory? \textcolor{mygray}{ | 15:15}    
                
                \item [ Kroemer N.] How to design a good task: lessons from statistical and computational models of behavior and brain responses \textcolor{mygray}{ | 15:30}    
                
            \end{description} 
            \end{symposium}
            