
            \begin{symposium}
            {S28 - Central nervous biomarkers of stress and resilience in the lab and in everyday life: Predictions and considerations }
            {Gina-Isabelle Henze, Lara M.C. Puhlmann}
            {Freitag 16:15 - 17:45 | Ort TODO}
            {Research Division of Mind and Brain, Department of Psychiatry and Psychotherapy CCM, Charité Universitätsmedizin Berlin, Corporate Member of Freie Universität Berlin, Humboldt-Universität zu Berlin, and Berlin Institute of Health; Big Data Institute, Li Ka Shing Centre for Health Information and Discovery, Nuffield Department of Population Health, University of Oxford; Institute of Psychology, University of Regensburg;}
            Stress reactions are holistic phenomena characterized by psychological and physiological activation encompassing the brain and the endocrine system. Understanding how the brain reacts to acute and long-term stress is therefore central to biopsychological stress research. It promises to identify neural biomarkers to predict future psycho-physiological stress reactions. However, biopsychological stress measures often do not correspond empirically. Our symposium discusses novel approaches to study how neural activation and plasticity correspond with stress and resilience trajectories in the laboratory and everyday life.
First, Gina-Isabelle Henze presents data from a mega-analysis including 500 subjects exposed to ScanSTRESS. For a subsample, it was further investigated if structure- and task-based brain measures can predict response trajectories of acute stress processing from baseline through acute to recovery phase.
Next, Peter Kirsch speaks about effects of autonomy support and physical activity of pupils (fifth and sixth graders) on their neural and cortisol stress responses as well as on brain development in the context of an education outside the classroom intervention.
Marina Giglberger reports on associations between acute neural stress responses and depression- and anxiety-symptoms as well as on the predictive value of these neural correlates for the course of depression- and anxiety-symptom measures in healthy subjects in daily life (over 13 months).
Lara Puhlmann then discusses how psychological stress reactions as a proxy for mental resilience can be measured and predicted in cross-sectional as well as longitudinal studies.
Finally, Meike Hettwer presents data on how longitudinal trajectories of resilience are related to progressive cortical myelination during adolescence.
            \begin{description}    
            
                \item [ Henze G.] The brain under acute stress: Triple network reactions and prediction of psycho-endocrine response trajectories \textcolor{mygray}{ | 16:15}    
                
                \item [ Kirsch P.] Choice and movement matters: Pupils' stress regulation, brain development and brain function in an outdoor education project  \textcolor{mygray}{ | 16:30}    
                
                \item [ Giglberger M.] The association between neural stress responses and symptoms of anxiety and depression \textcolor{mygray}{ | 16:45}    
                
                \item [ L.M.C. P.] Resilience quantification via psychological stressor reactivity scores \textcolor{mygray}{ | 17:00}    
                
                \item [ Hettwer M.] Longitudinal trajectories of resilient psychosocial functioning link to ongoing cortical myelination and functional reorganization during adolescence \textcolor{mygray}{ | 17:15}    
                
            \end{description} 
            \end{symposium}
            