
            \begin{symposium}
            {S21 - The German National Cohort (NAKO) as a resource for mental health research}
            {Maja P. Völker, Fabian Streit}
            {Freitag 09:00 - 10:30 | Room 0.12/0.13}
            {Department of Genetic Epidemiology in Psychiatry, Central Institute of Mental Health, Medical Faculty Mannheim, Heidelberg University; Department for Psychiatry and Psychotherapy, Central Institute of Mental Health, Medical Faculty Mannheim, Heidelberg University; Hector Institute for Artificial Intelligence in Psychiatry, Central Institute of Mental Health, Medical Faculty Mannheim, Heidelberg University}
            The German National Cohort (NAKO) is a population-based prospective cohort study that investigates common diseases and their risk and protective factors. It is the largest German health study with 205,415 subjects aged 19-74 years recruited in 18 study centres. The symposium targets researchers who might benefit from working with NAKO data, and showcases its potential to investigate mental health, risk factors, and neural correlates. The first presentation provides an overview of the NAKO with a focus on psychiatric phenotypes. It will present the assessment strategy and show how researchers can access the data. Moreover, an overview of the instruments to assess mental health and related constructs, and of observed prevalences and associations with established risk factors is given. The second presentation shows how this dataset can be used to investigate risk factors for mental health. Findings on individual and joint effects of family history and childhood tRooma on depression are be presented. The third presentation gives an overview of the brain imaging performed in a subset of 30,868 individuals and introduces preliminary findings regarding associations of neuroimaging metrics with socio-demographic variables and cognitive domains, as well as the application of data in a deep learning-based brain-age model. The fourth presentation shows how structural brain imaging data can be combined with psychosocial factors to predict measures of anxiety and panic disorder. The applied machine learning algorithms showed good classification performance, the predictive power of psychosocial factors and highlighted the left amygdala as a relevant brain region.
            \begin{description}    
            
                \item [ Streit F.] An introduction to the German National Cohort (NAKO) with a focus on psychiatric phenotypes \textcolor{mygray}{ | 09:00}    
                
                \item [ Völker M.] Individual and Joint Effects of Family History of Depression and Childhood Trauma on Current and Lifetime Depression \textcolor{mygray}{ | 09:20}    
                
                \item [ Jockwitz C.] Overview of the (brain) MRI assessment in the NAKO \textcolor{mygray}{ | 09:40}    
                
                \item [ Gutzeit J.] Classification of anxiety and panic using structural MRI data and psychosocial factors: machine learning results from the NAKO study \textcolor{mygray}{ | 10:00}    
                
            \end{description} 
            \end{symposium}
            