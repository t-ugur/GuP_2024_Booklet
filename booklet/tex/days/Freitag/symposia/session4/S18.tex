
            \begin{symposium}
            {S18 - Exploring the Layers of Language Prediction: From Phonemes to Paragraphs}
            {Merle Schuckart, Sandra Martin}
            {Freitag 09:00 - 10:30 | Room 1.12/1.13}
            {University of Lübeck; Max Planck Institute for Human Cognitive and Brain Sciences}
            Language prediction plays a pivotal role in understanding and facilitating everyday communication. It operates on multiple levels and time scales, enabling us to anticipate everything from phonemes and syllables to words, meanings, and syntactic structures. Each granularity level contributes uniquely to our ability to comprehend language, making communication more seamless. This symposium illustrates the breadth of the methodological intricacies of language prediction research and how predictability shapes language comprehension and production across different time scales.
Firstly, Peter Donhauser presents two MEG studies on prediction during natural listening at the phonemic time-scale, highlighting predictions at the most granular levels. Jill Kries then shows how the neural dynamics of phoneme representation interact with lexical predictability, in healthy participants and individuals with aphasia. Moreover, she will also present her ongoing intracranial EEG work on the decoding of speech features such as word predictability, during speech comprehension and production. Following this, Merle Schuckart shares findings from a behavioral self-paced reading experiment, illustrating the influence of increased cognitive load on language prediction across several time scales, and how this relationship is modulated by cognitive aging. Lastly, Cas Coopmans discusses the role of syntactic structure building in natural language comprehension. Using MEG data, he provides novel evidence for predictive structure building during story listening.
We envisage a controversial and fruitful discussion of conceptual and methodological links between these approaches. How might these diverse perspectives on language prediction reshape our understanding of communication? Join us in exploring these innovative studies.
            \begin{description}    
            
                \item [ Donhauser P.] Neurophysiology of speech predictions at the sublexical level \textcolor{mygray}{ | 09:00}    
                
                \item [ Kries J.] How lexical predictability affects neural dynamics of phoneme representations in neurotypicals and individuals with aphasia \textcolor{mygray}{ | 09:20}    
                
                \item [ Schuckart M.] Contribution of cognitive control resources to natural language comprehension across the adult lifespan \textcolor{mygray}{ | 09:40}    
                
                \item [ Coopmans C.] Predicting syntactic structures during naturalistic language comprehension \textcolor{mygray}{ | 10:00}    
                
            \end{description} 
            \end{symposium}
            