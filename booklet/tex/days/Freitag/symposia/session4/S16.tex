
            \begin{symposium}
            {S16 - Multidisciplinary and multimodal perspectives on episodic memory in neuropsychiatric disorders}
            {Jessica Peter, Michael Orth }
            {Freitag 09:00 - 10:30 | Ort TODO}
            {University of Luebeck}
            The ability to form and retrieve memories about personal experiences is paramount for human existence. The quantity and quality of such memories depend on numerous factors to do with the experiences themselves but also with the complexities of the neuroanatomical basis of memory formation and retrieval. Much insight can be gained from deficits in this ability in the context of neuropsychiatric disorders. In this symposium, we will explore the topic of episodic memory formation and retrieval from different perspectives. We will look at the impact of ageing, degeneration, or surgical lesions on the hippocampus as a key hub in networks subserving episodic memory. We will then examine the influence mood states can have on formation and retrieval of emotionally-valenced memories and, vice-versa, how mood states may self-perpetuate because of what is being remembered.
This symposium will provide a neuroscience perspective on factors that influence episodic memory performance and its underlying neuroanatomy across the lifespan and in the context of neuropsychiatric conditions such as Alzheimer’s disease, Depression, Trauma, and Epilepsy. There will be five talks, each presenting cutting-edge research combining behavioural data with physiology or neuroimaging in different age groups or psychiatric conditions. We will discuss implications and possible directions for our understanding of episodic memory and future theoretical and experimental approaches that could be useful to fill the many remaining knowledge gaps.
            \begin{description}    
            
                \item [ Bunzeck N.] Trajectories and contributing factors of neural compensation in healthy and pathological ageing \textcolor{mygray}{ | 09:00}    
                
                \item [ Reber T.] Single neuronal mechanism of transitive inference: insights from invasive recordings in the human medial temporal lobe of epilepsy patients \textcolor{mygray}{ | 09:15}    
                
                \item [ Staniloiu A.] Dissociative Amnesia – A survey of 95 cases \textcolor{mygray}{ | 09:30}    
                
                \item [ Kobelt M.] Exploring neural representations during trauma-analog experiences and memory intrusions \textcolor{mygray}{ | 09:45}    
                
                \item [ Orth M.] Left DLPFC modulation induces cognitive reorganisation in patients with depression \textcolor{mygray}{ | 10:00}    
                
            \end{description} 
            \end{symposium}
            