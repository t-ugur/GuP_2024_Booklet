
            \begin{symposium}
            {S03 - Cognitive Brain States from a Network Neuroscience Perspective }
            {Session chair(s): Kirsten Hilger}
            {Donnerstag 10:30 - 12:00 | Ort TODO}
            {Welche Uni TODO}
            Autoren Kirsten Hilger, Sebastian Markett, Jonas Thiele, Urs Braun, Stephan Krohn, Anni Richter, Sofie Valk

Brain states are recurring patterns of distributed brain activity. Such states originate from the brain network's structural scaffold and offer insight into the complex operations enabling human cognition. Understanding their spatio-temporal properties, their occurrences as well as their transitions holds promise for deciphering the neurobiological bases of cognitive processes, individual differences and disease-specific alterations.
This symposium focuses on brain states arising from specific cognitive demands and discusses their potential influence on behavior. We showcase five contributions that employ experimental paradigms to explicitly trigger brain states of active cognition or provide an environmental context for interpreting the behavioral implications of brain state changes. All contributions utilize techniques from network science or artificial intelligence (AI) as common methodological framework.
We explore various aspects of brain states, beginning with states emerging from effortful cognitive processing during an established intelligence test. Next, the utilization of recurrent neural networks to simulate brain state dynamics during different cognitive tasks will be explored and the “complexome” will be introduced. Finally, memory-related brain states and their alterations during cognitive aging will be presented, and the examination of changes in brain states following cognitive training interventions will be outlined in the context of clinical applications.
Following these individual contributions, we will conclude the symposium with a podium discussion. This discussion will focus on opportunities and challenges induced by the study of brain states with techniques of network science and AI by considering conceptual and methodological boundaries.
            \begin{description}    
            
                \item [Jonas A. Thiele (TODO: refactor name)] TODO: add title \textcolor{mygray}{ | 10:30}    
                
                \item [Oliver Frank  (TODO: refactor name)] TODO: add title \textcolor{mygray}{ | 10:45}    
                
                \item [Stephan Krohn (TODO: refactor name)] TODO: add title \textcolor{mygray}{ | 11:00}    
                
                \item [Anni Richter (TODO: refactor name)] TODO: add title \textcolor{mygray}{ | 11:15}    
                
                \item [Sofie L Valk (TODO: refactor name)] TODO: add title \textcolor{mygray}{ | 11:30}    
                
            \end{description} 
            \end{symposium}
            