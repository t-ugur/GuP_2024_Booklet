
            \begin{symposium}
            {S01 - Exploring new approaches to increase utility of psychophysiological markers for clinical psychology}
            {Session chair(s): Julia Klawohn, Hannes Per Carsten}
            {Donnerstag 10:30 - 12:00 | Ort TODO}
            {Welche Uni TODO}
            Autor Julia Klawohn, Co-Autor: Hannes Per Carsten,

In recent years, biological markers have increasingly been incorporated into clinical research, showing they can further our knowledge of pathophysiological mechanisms underlying mental health problems and help identify new vantage points for interventions. Yet, despite these advances, limitations regarding robustness and effect sizes of associations between biological processes and clinical phenotypes remain and hamper practical utility of existing findings. This symposium will cover several studies that all attempt to improve bridging this gap by applying innovative experimental approaches or interventions, combining measures or methods, exploring underlying dimensions, or incorporating individualized materials. Kai Härpfer will present a large study spanning clinical and subclinical individuals, highlighting the importance of familial risk and lifetime diagnoses for variations in error-monitoring ERPs. Rosa Grützmann will show effects of a novel cognitive training aimed at reducing overactive error-processing in obsessive-compulsive disorder. She further demonstrates that a combination of several ERP markers can help improve diagnostic classification of mental health issues, like problematic internet use. Hannes Per Carsten extends the modification approach showing that VR-induced checking behavior may increase error-processing. Mareike Bayer presents a study combining EEG and fMRI to investigate emotional face processing in autism spectrum conditions, with results highlighting the importance of individualized stimuli. Finally, Julia Klawohn will show results indicating that emotional reactivity, as captured with both ERP and cardiac measures, can contribute to predictions of individual psychotherapy success in obsessive-compulsive disorder. Together, these findings capture the complexity of links between psychophysiological markers and psychopathology and propose leverage points for translational research strategies.
            \begin{description}    
            
                \item [Kai Härpfer (TODO: refactor name)] TODO: add title \textcolor{mygray}{ | 10:30}    
                
                \item [Rosa Grützmann (TODO: refactor name)] TODO: add title \textcolor{mygray}{ | 10:45}    
                
                \item [Hannes Per Carsten (TODO: refactor name)] TODO: add title \textcolor{mygray}{ | 11:00}    
                
                \item [Mareike Bayer (TODO: refactor name)] TODO: add title \textcolor{mygray}{ | 11:15}    
                
                \item [Julia Klawohn  (TODO: refactor name)] TODO: add title \textcolor{mygray}{ | 11:30}    
                
            \end{description} 
            \end{symposium}
            