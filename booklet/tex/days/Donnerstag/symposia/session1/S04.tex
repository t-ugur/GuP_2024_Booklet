
            \begin{symposium}
            {S04 - Perception under uncertainty}
            { Franziska Knolle}
            {Donnerstag 10:30 - 12:00 | Room 0.12/0.13}
            {University of Marburg }
            In recent years, there has been a growing understanding of the brain as a predictive entity (Friston, 2010; Clark, 2013; Yon \& Frith, 2021). Given the often noisy and ambiguous nature of sensory input, coupled with its indirect accessibility, our perception of the environment is covered in uncertainty. Consequently, the brain must construct a model of the world to anticipate or predict future events and infer their causes. Formalised in the Bayesian brain hypothesis, prior knowledge is combined with sensory likelihood to generate percepts, and prediction error - the difference between what has been expected based on the prior and the sensory input - are used to update the model and minimise prediction errors in the future (Rascola \& Wagner, 1972). However, due to the probabilistic nature of the prior knowledge and sensory likelihood, the model must be flexible enough to tolerate inaccuracies and yet adaptable to change, despite the relative uncertainty of prediction errors. Interestingly, imbalances between precision of prior knowledge and precision of sensory likelihood may result in false percepts and may even explain the emergence of clinical symptoms, such as hallucinations or delusions. While this may explain how the brain reacts to uncertainty, the underlying mechanisms still remain poorly understood, requiring empirical evidence.
With this symposium, we are, therefore, bringing together recent evidence from different experimental studies using behaviour, TMS, EEG, MEG, and computational modelling in clinical and non-clinical samples to investigate how uncertainty impacts perception within different modalities in clinical and non-clinical populations.
            \begin{description}    
            
                \item [ Eckert A.] Cross-Modality Evidence for Reduced Choice History Biases in Psychosis-Prone Individuals  \textcolor{mygray}{ | 10:30}    
                
                \item [ Peylo C.] Oscillatory signatures of predictions in social and sensory perception  \textcolor{mygray}{ | 10:45}    
                
                \item [ Haarsma J.] Shared and diverging neural dynamics underlying false and veridical perception  \textcolor{mygray}{ | 11:00}    
                
                \item [ Sterner E.] Alterations in predictive language processing are associated with schizotypal and autistic traits  \textcolor{mygray}{ | 11:15}    
                
                \item [ Knolle F.] Alterations in the use of prior semantic knowledge relative to sensory information from acute psychosis to psychotic remission: a longitudinal approach  \textcolor{mygray}{ | 11:30}    
                
            \end{description} 
            \end{symposium}
            