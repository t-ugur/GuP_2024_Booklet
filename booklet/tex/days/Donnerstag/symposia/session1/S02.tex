
            \begin{symposium}
            {S02 - Leveraging Single-Trial Electrophysiological Data in the Cognitive Neurosciences: Implementations, Insights, and Challenges}
            {Session chair(s): Norman Forschack}
            {Donnerstag 10:30 - 12:00 | Ort TODO}
            {Welche Uni TODO}
            Autoren: Norman Forschack, Christopher Gundlach, Alina Studenova, Martina Kopcanova, Christian Keitel, Niels A. Kloosterman

Electrophysiological recordings have been instrumental in unraveling the brain's functional dynamics, offering insights into neural processes underlying perception, cognition, and behavior. In recent years, there has been a growing interest in the analysis of single-trial electrophysiological data, which holds the promise of unveiling the brain's intricate temporal dynamics on a trial-by-trial basis. The symposium aims to explore the wealth of knowledge that can be derived from investigating single-trial data and to address the unique challenges that researchers face in harnessing its potential.
Specifically, the symposium asks how utilizing single-trial data helps to characterize and conceptualize general principles of neural processing and what we can learn about general neural mechanisms underlying perception, cognitive functions, and behavior. The work discussed in the symposium will focus on the analysis of neural measures such as ‘traditional’ event-related potentials (ERPs), oscillatory activity, and ‘new’ measures like signal entropy as well as single-trial behavioral measures. By showcasing recent experiments, we will explore emerging technologies and computational tools that can aid in the analysis of single-trial data, such as spectral decomposition, denoising procedures, information theoretic measures, and Bayesian statistical modeling approaches. We present how these novel methodologies are implemented to explicitly probe and uncover general neural mechanisms underlying perception and cognitive functions. In a joint discussion, we will deepen our understanding of the advantages, limitations, conceptual assumptions, and prerequisites as well as the obstacles of analyzing single-trial data to advance our understanding of the dynamic processes of the brain and their relationship to human cognition and behavior.
            \begin{description}    
            
                \item [Alina Studenova  (TODO: refactor name)] TODO: add title \textcolor{mygray}{ | 10:30}    
                
                \item [Christopher Gundlach (TODO: refactor name)] TODO: add title \textcolor{mygray}{ | 10:50}    
                
                \item [Christian Keitel (TODO: refactor name)] TODO: add title \textcolor{mygray}{ | 11:10}    
                
                \item [Niels A. Kloosterman (TODO: refactor name)] TODO: add title \textcolor{mygray}{ | 11:30}    
                
            \end{description} 
            \end{symposium}
            