
            \begin{symposium}
            {S09 - The focused mind: neural signatures of selective attention in perception and memory}
            {Melinda Sabo, Daniel Schneider }
            {Donnerstag 13:00 - 14:30 | Large auditorium}
            {Leibniz Research Centre for Working Environment and Human Factors (IfADo) at the Technical University Dortmund; Leibniz Research Centre for Working Environment and Human Factors (IfADo) at the Technical University Dortmund}
            Selective attention plays a crucial role in shaping both perception and memory. Here, we present recent studies that investigate this role at the levels of perception, working memory and long-term memory using different neuroimaging approaches. Thereby, we aim to characterize selective attention and its neural signatures at different stages of information processing, and how they relate to goal-directed behavior. In the first talk, Sarah Tune will provide evidence for the neural implementation of auditory attention via neural speech tracking and its functional relevance to attentive listening behavior in a longitudinal cohort of aging individuals. Next, Niko Busch will discuss whether alpha power lateralization (~10 Hz) plays a causal role in attentional orienting in the context of perception and working memory. The following two talks will address the topic of distraction during working memory storage. First, Philipp Deutsch will show how unattended and attended distractors interfere with neural representations of auditory content held in working memory captured via decoding of fMRI signals. Subsequently, Daniel Schneider will demonstrate how oscillatory EEG correlates of selective attention can be used to investigate the resumption of a working memory task after an interruption. Finally, based on EEG findings, Melinda Sabo will discuss whether the principles of attentional selection obtained in the perceptual and working memory domain can be transferred to long-term memory. This presentation will conclude the symposium by examining the similarities and differences in selective attention for various instances of goal-directed information processing.
            \begin{description}    
            
                \item [ Tune S.] Can neural attentional filters predict listening behaviour dynamics in healthy aging? \textcolor{mygray}{ | 13:00}    
                
                \item [ Busch N.] The role of lateralized alpha oscillations in visual exogenous attention and short-term memory \textcolor{mygray}{ | 13:15}    
                
                \item [ Deutsch P.] Distraction Disrupts Working Memory Decoding in Auditory Cortex \textcolor{mygray}{ | 13:30}    
                
                \item [ Schneider D.] Interrupting working memory: A self-paced resumption phase facilitates primary task performance following an interruption \textcolor{mygray}{ | 13:45}    
                
                \item [ Sabo M.] The spotlight of memory: attentional selection of internal long-term memory representations \textcolor{mygray}{ | 14:00}    
                
            \end{description} 
            \end{symposium}
            