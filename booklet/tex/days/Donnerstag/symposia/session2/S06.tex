
            \begin{symposium}
            {S06 - Facets of (a)motivation: effort-based decision-making in health and disease }
            {Session chair(s): Mario Bogdanov}
            {Donnerstag 13:00 - 14:30 | Ort TODO}
            {Welche Uni TODO}
            Autoren: Mario Bogdanov, Dennis Hernaus, Manuel Kuhn, Kristína Pavlíčková, Matthias Pillny, Corinna Schulz

Motivation constitutes an essential requirement for any form of goal-directed behavior. Yet, people’s willingness to perform actions for desired outcomes varies substantially and can be affected by a plethora of intra-personal and environmental factors, such as the individual’s affective state or the presence of psychiatric or neurological conditions. However, many of the behavioral and biological mechanisms governing motivated behavior remain elusive. To investigate motivational processes more formally, recent work has started to employ effort-based decision-making tasks grounded in the (neuroeconomic) assumption that individuals modulate effort investment based on a cost-benefit analysis that weighs anticipated task demand against potential rewards. In this symposium, we will present data from recent studies on effort-based decision-making in clinical and healthy populations, highlighting specific motivational deficits observed in current and remitted psychopathology, diverse modulatory influences of stress exposure on motivation, and novel avenues to treatment of amotivation. First, Matthias Pillny presents meta-analytic findings on effort-based decision-making in patients with depressive disorders and schizophrenia, showcasing the transdiagnostic nature of these symptoms. Then, Manuel Kuhn provides evidence for impaired decision-making processes in patients with remitted depression based on computational modeling of choice behavior. Mario Bogdanov will link both recent stress exposure and early-life adversity to reduced effort exertion for reward in the present. In contrast, Kristína Pavlíčková and Dennis Hernaus showcase how acute stress may increase effort expenditure to avoid punishment. Finally, Corinna Schulz will present how manipulations of the body-brain axis may help to increase motivation in patients with major depression.
            \begin{description}    
            
                \item [Matthias Pillny  (TODO: refactor name)] TODO: add title \textcolor{mygray}{ | 13:00}    
                
                \item [Manuel Kuhn  (TODO: refactor name)] TODO: add title \textcolor{mygray}{ | 13:15}    
                
                \item [Mario Bogdanov (TODO: refactor name)] TODO: add title \textcolor{mygray}{ | 13:30}    
                
                \item [Kristína Pavlíčková (TODO: refactor name)] TODO: add title \textcolor{mygray}{ | 13:45}    
                
                \item [Corinna Schulz (TODO: refactor name)] TODO: add title \textcolor{mygray}{ | 14:00}    
                
            \end{description} 
            \end{symposium}
            