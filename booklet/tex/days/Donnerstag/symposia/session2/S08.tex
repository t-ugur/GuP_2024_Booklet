
            \begin{symposium}
            {S08 - Unraveling cognitive and executive functions using human single-neuron recordings}
            {Jonathan Daume, Stefanie Liebe}
            {Donnerstag 13:00 - 14:30 | Small auditorium}
            {Netherlands Institute for Neuroscience}
            Single-neuron recordings provide unparalleled insights into neural mechanisms underlying human behavior at cellular resolution. This symposium showcases several lines of research from intracranial recordings in humans that reveal the critical role of single-neuron activity in cognitive and executive functions. Our studies span the dynamic neural underpinnings of language comprehension, the processes underlying memory encoding and control, and the neural disruptions of motor and cognitive aspects in movement disorders. The first talk reveals how pronouns reactivate specific neuron representations of nouns, emphasizing a dynamic semantic memory network crucial for efficient language comprehension. The second presentation challenges a traditional view on the neural implementation of temporal order memory and provides a novel link between sequential memory and stimulus timing using recurrent neural network modeling. The third study explores the regulation of working memory through theta-gamma phase-amplitude coupling, demonstrating how cognitive control and hippocampal single-neuron activity converge to enhance memory fidelity. The final talk uncovers the critical role of a cognitive-motor basal ganglia interface in locomotion control in Parkinson's disease. Through these diverse yet interconnected studies, the symposium demonstrates the unique contributions of human single-neuron recordings to our understanding of the brain's capability to process complex cognitive tasks in language, memory and movement control, and advances our fundamental knowledge of the neural underpinnings underlying human cognition.
            \begin{description}    
            
                \item [ Dijksterhuis D.] Pronouns activate concept cells in the human hippocampus \textcolor{mygray}{ | 13:00}    
                
                \item [ Liebe S.] Theta-based spike-phase coding supports temporal-order working memory in the human MTL and recurrent neural networks \textcolor{mygray}{ | 13:20}    
                
                \item [ Daume J.] Control of working memory by phase-amplitude coupling of human hippocampal neurons  \textcolor{mygray}{ | 13:40}    
                
                \item [ Gulberti A.] Neuronal signals from deep brain areas related to the freezing of gait phenomenon in Parkinson’s disease \textcolor{mygray}{ | 14:00}    
                
            \end{description} 
            \end{symposium}
            