
            \begin{symposium}
            {S07 - Exploring Methodological Challenges and Solutions in Psychophysiological Research: The case example of EEG}
            {Johannes Rodrigues}
            {Donnerstag 13:00 - 14:30 | Ort TODO}
            {Johannes Gutenberg University}
            This symposium delves into challenges around validity, reliability, and interpretability of findings while analyzing psychophysiological data. A range of methodological issues will be explored using electroencephalography (EEG) as an illustrative example.
First, Mareike Hülsemann discusses the approach of mass univariate analysis with cluster-based permutation testing (Groppe et al., 2011). The advantages and pitfalls of this data-driven approach are illustrated using the example of event-related and time-frequency analysis in the auditory and visual domains.
Second, Mario Reutter investigates the influence of analysis decisions on the trade-off between effect size and reliability of the N2pc component in a Dot Probe paradigm. The results are integrated into a multi-level perspective on signal-to-noise ratios and related to the reliability paradox (Hedge et al., 2018).
Third, Johannes Rodrigues will present the impact of quantification methods and reference schemes (CSD, linked mastoids, average) on feedback-related negativity (FRN) amplitudes in a trust game paradigm: In addition, the data questions the quality criterion SME provided by Luck et. al., 2021.
Fourth, Sven Lesche will introduce a template matching algorithm that can automatically extract ERP component latencies and provides a fit statistic quantifying the degree of certainty in measurement. Results from a simulation study aiming to validate this new approach will be discussed.
In summary, this symposium fosters dialogue and innovation to overcome methodological challenges in psychophysiological research.
            \begin{description}    
            
                \item [ Hülsemann M.] Chances and pitfalls of mass univariate analysis with cluster-based permutation testing for exploratory and hypothesis-driven psychophysiological research  \textcolor{mygray}{ | 13:00}    
                
                \item [ Reutter M.] The trade-off between effect size and reliability: Insights from a multiverse-analysis of electroencephalographical data within a Dot Probe paradigm. \textcolor{mygray}{ | 13:20}    
                
                \item [ Rodrigues J.] Questioning the suitability of the quality criterion SME for EEG data: Exploration of the influence of the quantification method and reference scheme on FRN and SME of FRN amplitudes. \textcolor{mygray}{ | 13:40}    
                
                \item [ Lesche S.] Automatically Extracting ERP Component Latencies Using a Dynamic Template Matching Algorithm \textcolor{mygray}{ | 14:00}    
                
            \end{description} 
            \end{symposium}
            