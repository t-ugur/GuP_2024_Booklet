
            \begin{symposium}
            {S10 - Towards the Study of Interacting Emotional and Cognitive Processes and Their Neurophysiological Basis}
            {Katharina Lingelbach}
            {Donnerstag 13:00 - 14:30 | Room 0.12/0.13}
            {Fraunhofer IAO, Unversity of Oldenburg}
            In the past, cognitive and emotional processes were often investigated separately, ignoring their interdependence in human nature. However, there has been a recent shift towards studying how these processes interact. Nevertheless, the underlying neurophysiological basis and mechanisms remain unclear.
The symposium aims to integrate new insights from various approaches and paradigms. It will be organised around three research questions. How and in what circumstances do cognitive states, like attention or load, modulate emotional processing (R1)? How do emotional stimuli affect cognitive processes, such as memory, attention or executive functioning (R2)?
Utilizing intracranial recordings of the amygdala and parallel scalp recordings, Enya Weidner shows how attention to valence tunes emotion processing in the face processing network and amygdala and how this interaction changes over time. Katharina Lingelbach presents spatiotemporal and oscillatory signatures of simultaneous and sustained dual-task interactions of emotional face processing and working memory load. Anya Dietrich introduces temporal and frequency-specific signatures of emotional interference inhibition and emotion-cognition integration.
How does the brain regulate emotional processing or alter the meaning of emotional stimuli (R3)?
Christoph Scheffel discusses the role of cognitive effort in emotion regulation. Effects of effort and emotion regulation in different task phases and post-regulatory effects will be addressed on a subjective and physiological level. Maren Bertheau talks about error potentials and their link to emotion regulation when monitoring moral decisions made by autonomous cars in a dilemma situation.
We close the symposium with an open discussion on the implications, challenges, and future directions of the research.
            \begin{description}    
            
                \item [ Weidner E.] How Attention to Valence tunes Widespread Emotion Processing: Insights from Intracranial EEG and Scalp Recordings \textcolor{mygray}{ | 13:00}    
                
                \item [ Lingelbach K.] Spatiotemporal and Oscillatory Signatures of Emotional Face Processing and Working Memory Load in a Dual-Task \textcolor{mygray}{ | 13:15}    
                
                \item [ Dietrich A.] Understanding the Neural Mechanisms of Emotion-Cognition Interaction in Space, Time, Frequency, And Information Transfer \textcolor{mygray}{ | 13:30}    
                
                \item [ Scheffel C.] The Role of Cognitive Effort in Emotion Regulation \textcolor{mygray}{ | 14:00}    
                
                \item [ Bertheau M.] The Moral Machine in the EEG Lab \textcolor{mygray}{ | 14:15}    
                
            \end{description} 
            \end{symposium}
            