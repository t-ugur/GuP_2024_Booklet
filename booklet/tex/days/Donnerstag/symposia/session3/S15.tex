
            \begin{symposium}
            {S15 - Double the trouble, twice the fun: interactions between multiple mental representations across the cortical sheet.}
            {Thomas Christophel}
            {Donnerstag 14:45 - 16:15 | Room 1.18}
            {Humboldt University Berlin}
            As we navigate daily life, our senses are subject to a merciless onslaught of incoming stimulation only a tiny subset of which is relevant for future behavior. Working memory enables us to retain these relevant inputs for subsequent use. But our cortical representational resources are limited and consequently we are tasked with encoding, updating, selecting, removing, and refreshing the right information using these resources. Work focusing on individual mental representations and their cortical instantiation has revealed that both anterior and posterior regions can represent working memory contents. Here, using data from humans and nonhuman primates, we investigate the interactions between multiple mental and neural representations using multivariate pattern analysis. Rosanne Rademaker will show how incoming sensory information interacts with items held in working memory, and what happens when visual input becomes task relevant. Thomas Christophel will demonstrate how load effects in delayed recall can be explained by altered recruitment of cortical regions as the number of retained items increases. Polina Iamshchinina will talk about how attending to a sensory input and selecting a mental representation from working memory relies on a shared mechanism. Surya Gayet will interrogate bidirectional interactions between memory representations and perception during naturalistic search. Finally, Christoph Blewdowski will demonstrate that even item from previous trials that are inadvertently represented can alter currently relevant mental representations and behavior. Thereby we shed light on the manifold of interactions between concurrently represented mental representations and how they jointly guide our interactions with the world around us.
            \begin{description}    
            
                \item [ Rademaker R.] Manipulating attentional priority creates a trade-off between memory and sensory representations in human visual cortex  \textcolor{mygray}{ | 14:45}    
                
                \item [ Christophel T.] Independent representational roles for sensory and anterior regions under working memory load  \textcolor{mygray}{ | 15:00}    
                
                \item [ Iamshchinina P.] Neural ensembles within prefrontal cortex generalize across attention and working memory  \textcolor{mygray}{ | 15:15}    
                
                \item [ Gayet S.] Bidirectional interactions between memory representations and perception during naturalistic search \textcolor{mygray}{ | 15:30}    
                
                \item [ Bledowski C.] A direct neural signature of serial dependence in working memory \textcolor{mygray}{ | 15:45}    
                
            \end{description} 
            \end{symposium}
            