
            \begin{symposium}
            {S15 - Psychobiology of Treatment Expectation }
            {Session chair(s):  Lieven Schenk, Stefanie Brassen}
            {Donnerstag 16:30 - 18:00 | Ort TODO}
            {Welche Uni TODO}
            Autoren: Lieven Schenk, Stefanie Brassen

This symposium will delve into latest advances in research on individuals’ treatment expectations as important modulators of health outcomes. Understanding the psychobiological mechanisms of this influence has the potential to capitalize on these effects, optimizing treatment strategies and improving health outcomes. Early career researchers from the Collaborative Research Center "Treatment Expectation" (CRC 289) will provide novel neurobehavioral insights into the effects of positive (placebo) and negative (nocebo) expectations on pain and the affective system. The presented studies used controlled induction protocols and a rich variety of methods, including neuroimaging, psychophysiology, behavioural measurements, and machine learning approaches.
First, Lieven Schenk will present data on the amplification of treatment expectations and placebo analgesia through negative side effects, indicating a strong involvement of the descending pain modulatory system. Second, Jana Aulenkamp will address the differences in expectation effects on visceral and somatic pain perception, emphasizing the role of negative instruction and experience. Next, Christoph Wittkamp will present the influence of positive and negative expectations on pain perception using EEG-fMRI, highlighting distinct neural representations during pain processing. Afterwards, Daniela Marrero Polegre will demonstrate how positive expectations can improve mood and emotional processing in older individuals, with a special focus on prefrontal-limbic regulation. Finally, Raviteja Kotikalapudi will utilize machine learning approaches on large neuroimaging datasets to predict individual differences in treatment expectations, underlining the potential for personalized medicine and clinical interventions.
This symposium aims to promote interdisciplinary discussions about the principles and potential benefits of expectations in health and disease.
            \begin{description}    
            
                \item [Lieven A. Schenk (TODO: refactor name)] TODO: add title \textcolor{mygray}{ | 16:30}    
                
                \item [Jana L. Aulenkamp (TODO: refactor name)] TODO: add title \textcolor{mygray}{ | 16:45}    
                
                \item [Christoph Wittkamp (TODO: refactor name)] TODO: add title \textcolor{mygray}{ | 17:00}    
                
                \item [Daniela Marrero Polegre (TODO: refactor name)] TODO: add title \textcolor{mygray}{ | 17:15}    
                
                \item [Raviteja Kotikalapudi (TODO: refactor name)] TODO: add title \textcolor{mygray}{ | 17:30}    
                
            \end{description} 
            \end{symposium}
            