
            \begin{symposium}
            {S13 - Exploring emotional dynamics: Physiological and subjective insights from clinical and experimental perspectives }
            { Janine Wirkner, Maike Hollandt}
            {Donnerstag 14:45 - 16:15 | Large auditorium}
            {University of Greifswald}
            In this symposium, we explore research on threat reactivity, fear extinction, and emotion regulation, providing valuable insights from both clinical and experimental perspectives to enhance our understanding of the complex biopsychosocial factors associated with mental disorders.
The first contribution investigates threat reactivity in patients with primary anxiety or depressive disorders within the Research Domain Criteria (RDoC) framework. It highlights diverse threat reactivity patterns associated with clinical characteristics, emphasizing the necessity of comprehensive assessments in clinical settings. Utilizing an event-related approach, the second study explores fear extinction in patients with anxiety disorders compared to healthy controls. It suggests increased uncertainty among patients during extinction training, shedding light on potential mechanisms underlying fear extinction deficits. The third study investigates extinction generalization in exposure-based treatments, employing mental imagery to promote the updating of extinction memory. This approach shows promise in enhancing extinction for specific stimuli, offering new avenues for treatment development. Building on the defense cascade model, the fourth study examines autonomic defensive responses to social threat. It identifies similar dynamics in response to approaching social threat, while also uncovering specific response patterns not previously observed in research. In the final contribution, a novel paradigm combining emotional conflict paradigms with multimodal measurements is presented. This innovative approach underscores the potential for advancing our understanding of emotional regulation processes. Together, these studies contribute to a deeper comprehension of the mechanisms underlying mental disorders, offering insights that may inform both clinical practice and future research.
            \begin{description}    
            
                \item [ Hollandt M.] Individual differences in threat reactivity among patients with anxiety and depressive disorders based on the RDoC framework \textcolor{mygray}{ | 14:45}    
                
                \item [ Droste K.] Exploring the multimodal dynamics of threat expectancy change during fear extinction: Insights from a novel event-related approach \textcolor{mygray}{ | 15:00}    
                
                \item [ Scheuermann D.] Enhancing extinction generalization in a category-based fear conditioning paradigm \textcolor{mygray}{ | 15:15}    
                
                \item [ Szeska C.] Dynamic organization of autonomic defensive responses to social threat \textcolor{mygray}{ | 15:30}    
                
                \item [ Yang Y.] Multimodal measurement of Affective Expressive Flexibility (AEF) – Evaluation of a new experimental paradigm \textcolor{mygray}{ | 15:45}    
                
            \end{description} 
            \end{symposium}
            