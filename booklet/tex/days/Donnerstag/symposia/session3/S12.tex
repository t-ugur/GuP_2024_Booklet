
            \begin{symposium}
            {S12 - Uncovering lifespan signatures: A multimodal perspective at the interface of development, aging, and disease risk }
            {Christina Stier}
            {Donnerstag 14:45 - 16:15 | small auditorium}
            {nan}
            The emergence of various disorders often coincides with specific age windows, indicating alterations in developmental or aging pathways. Efforts to quantify biological aging and establish normative modeling using brain imaging have gained considerable momentum, mainly driven by big-data initiatives and MRI methods. This symposium aims to demonstrate the utility of such approaches and integrate evidence from functional modalities for a comprehensive understanding of the human lifespan.
Dominik Kraft (University of Tübingen) will focus on the variability of brain-puberty interactions and the problem of high-dimensional data and embedding. Then, Philippe Jawinski (Humboldt University Berlin) will present results from the largest genome-wide association study of structural brain age gaps to date, linking accelerated or decelerated aging to mental and physical health. Similar prediction analyses and age-related studies utilizing M/EEG have often been limited by methodological constraints in the past, resulting in rare or inconsistent findings. Christina Stier (University Hospital Münster) will address this gap by discussing conventional and novel markers of neural dynamics that are informative of individual age across adulthood. Related to this, Elena Cesnaite (University of Münster) will then elaborate on brain-cognition relationships in old age using a large EEG repository, focusing on (non-) rhythmic activity. Deniz Kumral (University of Freiburg) will close the circle by discussing the contributions of age-related signal variability obtained with fMRI and EEG and the interplay with the brain’s structural architecture.
Overall, a panel of emerging experts will provide multimodal perspectives on the lifespan, incorporating machine learning and imaging, genetics, cognition, and structure-function relationships.
            \begin{description}    
            
                \item [n n.] Investigating brain development through the lenses of pubertal maturation \textcolor{mygray}{ | 14:45}    
                
                \item [ Jawinski P.] Genome-wide analysis of brain age identifies 25 associated loci and unveils relationships with mental and physical health \textcolor{mygray}{ | 15:00}    
                
                \item [ Stier C.] Time-series phenotyping across the adult lifespan: age prediction using MEG and massive feature extraction \textcolor{mygray}{ | 15:15}    
                
                \item [ Cesnaite E.] Alterations in rhythmic and non‐rhythmic resting‐state EEG activity and their link to cognition in older age \textcolor{mygray}{ | 15:30}    
                
                \item [ Kumral D.] Linking structural and functional changes during aging \textcolor{mygray}{ | 15:45}    
                
            \end{description} 
            \end{symposium}
            