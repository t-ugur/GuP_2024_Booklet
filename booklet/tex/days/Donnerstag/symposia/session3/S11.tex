
            \begin{symposium}
            {S11 - Bridging the gap: a translational perspective on memory related sleep oscillations across species.}
            {Fabian Schwimmbeck, Jacqueline van der Meij}
            {Donnerstag 14:45 - 16:15 | Room 1.12/1.}
            {University of Tübingen}
            Sleep is thought to support memory consolidation, with particular sleep oscillations, such as cortical slow oscillations, thalamic spindles and hippocampal sharp-wave ripples, facilitating interregional communication and plasticity. Animal electrophysiology has traditionally spearheaded work on the neural scaffolding of memory consolidation. However, rare invasive recordings in humans now facilitate translating and extending these findings across species.
Following a translational perspective, this symposium will present the latest findings on ripples and sleep oscillations from animal and human research to provide a forum for cross-species discussion.
The first two talks will focus on sleep oscillations in rodents. Niels Niethard (University of Tübingen) will talk about the role of sleep oscillations in mediating network dynamics within local hippocampal and cortical circuits during SWS and subsequent REM sleep.
Jacqueline van der Meij (Radboud University Nijmegen) will present behavioural and electrophysiological results of a newly developed task to study the development of cognitive maps in rats.
In a translational approach, Frank van Schalkwijk (University of Tübingen) demonstrates a functional division between archicortical ripples promoting hippocampal-neocortical communication and neocortical ripples facilitating local cortical processes in both humans and rodents.
Finally, Fabian Schwimmbeck (University of Munich) will provide evidence for a ripple-triggered hippocampal-neocortical information flow by leveraging single-neuron recordings along the hippocampal output network during human sleep.
Together, this symposium showcases how sleep oscillations shape neural dynamics in rodents and humans, fostering discussions on converging evidence for their pivotal role in memory consolidation across species."
            \begin{description}    
            
                \item [ Niethard N.] Sleep oscillations and synaptic plasticity: A circuit perspective \textcolor{mygray}{ | 14:45}    
                
                \item [ Meij J.] Learning evoked brain activity at a local and global scale during sleep \textcolor{mygray}{ | 15:05}    
                
                \item [ Schalkwijk F.] An evolutionary conserved division-of-labor between archicortical and neocortical ripples organizes information transfer during sleep \textcolor{mygray}{ | 15:25}    
                
                \item [ Schwimmbeck F.] Single-neuron activity in the human MTL shows directed hippocampal-neocortical information flow during sharp-wave ripples during sleep \textcolor{mygray}{ | 15:45}    
                
            \end{description} 
            \end{symposium}
            