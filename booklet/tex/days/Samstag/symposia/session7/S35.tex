
            \begin{symposium}
            {S35 - Neurocognitive mechanisms of cognitive flexibility and attention allocation }
            {Laura Klatt, Anna-Lena Schubert}
            {Samstag 09:00 - 10:30 | Ort TODO}
            {Leibniz Research Centre for Working Environment and Human Factors}
            The symposium focuses on cognitive flexibility, a vital skill for navigating the complexities of dynamic environments. Despite its everyday importance, the neurocognitive mechanisms underpinning cognitive flexibility remain poorly understood. This symposium aims to bridge the gap between cognitive psychology and cognitive neuroscience, shedding light on these mechanisms.

The first talk by Laura-Isabelle Klatt will focus on the interplay between sensory modalities in a dynamic environment, requiring flexible shifts of spatial attention. Specifically, she will present behavioral and electrophysiological data demonstrating how sound localization and auditory spatial attention are influenced by bimodal odorant stimulation. Philipp Musfeld will follow with the second talk, exploring how working memory and long-term memory flexibly interact to optimize resource allocation within the limited capacity of our cognitive system. Specifically, using behavioral and electrophysiological data, his talk investigates how we learn regularities in our environment from repeated exposure and challenges the predominant view that such learning processes occur implicitly. In the third talk, Jan Göttmann will discuss findings from two novel working memory tasks. His talk will explore the interplay between computational model parameters indicative of working memory target and distractor processing, and the electrophysiological markers of attention allocation in complex span tasks. Lastly, Anna-Lena Schubert's talk will present insights from an individual differences study that examines the manifestation of cognitive flexibility in frontal midline theta connectivity and probes the interrelationships among these neural measures, their task-based variations, and general cognitive abilities.
            \begin{description}    
            
                \item [ Klatt L.] Odorant-induced Sound Localization Bias: Behavioral Evidence and Neurophysiological Correlates \textcolor{mygray}{ | 09:00}    
                
                \item [ Musfeld P.] Repetition Learning Depends on Explicit Retrieval from Episodic Memory: Evidence from Behavioral and Neuroimaging Studies \textcolor{mygray}{ | 09:20}    
                
                \item [ Göttmann J.] Simulate, Develop, Infer: Measuring Individual differences in Working Memory Processes \textcolor{mygray}{ | 09:40}    
                
                \item [ Schubert A.] Temporal Dynamics of Global Theta Connectivity in Relation to Intelligence: Evidence from Three Cognitive Flexibility Tasks \textcolor{mygray}{ | 10:00}    
                
            \end{description} 
            \end{symposium}
            