
            \begin{symposium}
            {S31 - Neuromodulation - current challenges in method optimization}
            {Thomas Dresler, Miroslava Jindrova}
            {Samstag 09:00 - 10:30 | Room 1.18}
            {Johannes Gutenberg University Medical Center, Leibniz Institute for Resilience Research}
            Neuromodulation is a promising field employing various techniques (e.g. electric/magnetic stimulation, neurofeedback) to modulate brain activity and brain states in order to improve clinical symptoms or to investigate specific functions of different brain regions. Although the initial results of its clinical application seem promising, optimization and personalization of these methods are needed to advance treatment. In this symposium, we will provide an overview of different neuromodulation methods that have been investigated as potential clinical treatments for various mental disorders. The main focus will be on current challenges in optimizing these methods, including their combination with various neuroimaging modalities to achieve personalization and enhance the intended clinical effects. First, Maximilian Lückel will talk about the personalization of transcranial magnetic and ultrasonic stimulation by combining them with MRI. Second, Magdalena Mischke will report on the use of transcranial direct current stimulation to alleviate post-COVID fatigue and how individual electrophysiological and immunological parameters can predict outcomes and potentially optimize stimulation. Third, Beatrix Barth will elaborate on different targeting approaches of the motor cortex during functional near-infrared spectroscopy neurofeedback (NF) and their effects on the underlying processes of NF learning. Finally, Miroslava Jindrová will give an overview of NF training for emotion regulation, compare different methods (functional magnetic resonance imaging and electroencephalography), feedback timings, and the use of mental strategies.
            \begin{description}    
            
                \item [ Lückel M.] Personalized precision neuromodulation by combining transcranial magnetic and ultrasonic stimulation with neuroimaging \textcolor{mygray}{ | 09:00}    
                
                \item [ Mischke M.] The influence of transcranial direct current stimulation on post-COVID fatigue: a comprehensive analysis of the electrophysiological and immunological influences \textcolor{mygray}{ | 09:20}    
                
                \item [ Barth B.] Exploring underlying mechanisms of real-time single region neurofeedback, functional connectivity neurofeedback and support vector machine neurofeedback \textcolor{mygray}{ | 09:40}    
                
                \item [ Jindrová M.] The way to optimization of neurofeedback training protocols for emotion regulation \textcolor{mygray}{ | 10:00}    
                
            \end{description} 
            \end{symposium}
            