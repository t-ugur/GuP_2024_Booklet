
            \begin{symposium}
            {S31 - From Brain Mapping to Behavior: Multimodal Insights into TMS Effects}
            { Ole Numssen, Sandra Martin}
            {Samstag 09:00 - 10:30 | Ort TODO}
            {MR Center of Excellence, Center for Medical Physics and Biomedical Engineering, Medical University of Vienna}
            This symposium on transcranial magnetic stimulation (TMS) addresses the urgent need to deepen our understanding of TMS effects on the brain, particularly given the substantial intra- and inter-individual variability in responses. This variability underscores a critical challenge in optimizing TMS for both neuroscience research and clinical applications. We will explore TMS from several complementary angles: its integration with fMRI and EEG, advances in electric field modeling to refine TMS focality and dosage, and assessments of its effects on cognitive function using behavioral and statistical methods.

Maria Vasileiadi will highlight the use of interleaved TMS-fMRI to elucidate how individual differences, cognitive states, and stimulation parameters influence TMS effects and offer ways to improve treatment efficacy. Sybren van Hoornweder will discuss the application of advanced forward models in TMS-EEG to improve our understanding of cortical excitability and the neural basis of TMS responses. Sandra Martin will critically evaluate the differential effects of different TMS protocols on cognitive control and executive functions, highlighting the strategic importance of protocol selection. Ole Numssen will present personalized approaches to TMS, using electric field modeling to tailor interventions based on individual neurophysiological profiles, addressing the need for individualized treatment strategies.

By weaving together insights from EEG, MRI, electric field considerations, and behavioral and statistical analyses, the symposium aims to promote a holistic understanding of TMS. This collaborative approach is critical to advancing TMS research and clinical practice, reducing response variability, and improving the precision and efficacy of neuromodulation techniques.
            \begin{description}    
            
                \item [ Vasileiadi M.] Variability in interleaved TMS­-fMRI responses related to individual factors and cognitive state \textcolor{mygray}{ | 09:00}    
                
                \item [ Hoornweder S.] Investigating TMS-Induced Electric Fields and Subsequent EEG Source Fields: Analysing the N15 TMS-EEG Peak Considering Dose and State Effects \textcolor{mygray}{ | 09:20}    
                
                \item [ Martin S.] Beyond Motor Effects: The Impact of TMS Protocol Selection on cognitive functions \textcolor{mygray}{ | 09:40}    
                
                \item [ Numssen O.] Beyond One-Size-Fits-All: Advancing TMS with Personalized Electric Field Modeling \textcolor{mygray}{ | 10:00}    
                
            \end{description} 
            \end{symposium}
            