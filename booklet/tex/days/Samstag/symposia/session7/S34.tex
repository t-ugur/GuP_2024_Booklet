
            \begin{symposium}
            {S34 - Using Genetics to Understand Pathways to Mental Disorders}
            {Fabian Streit}
            {Samstag 09:00 - 10:30 | Room 0.12/0.13}
            {Department of Epidemiology and Preventive Medicine, University of Regensburg}
            The symposium aims to demonstrate how large-scale genetic data can be used to identify pathways that might predispose individuals to mental disorders.
Julian Konzok uses genomic structural equation modeling to investigate causal risk factors for internalizing and externalizing mental disorders. He identifies childhood maltreatment as a universal risk factor while indicating alcohol consumption as a specific risk factor for the externalizing dimension, and physical activity as a specific protective factor for the internalizing dimension.
Javier Schneider Penate's study on the genetics of extinction learning used polygenic scores (PGS) in a well-characterized sample subjected to a fear conditioning paradigm. He shows that the functional connectivity between key brain regions mediates the relationship of genetic risk for anxiety disorders and PTSD with fear learning.
Philippe Jawinski presents results from the ENIGMA-EEG consortium, investigating genetic associations of resting-state EEG oscillations. Using data from nine cohorts and from up to 14,361 participants, the study demonstrates substantial SNP-based heritability, identifies associated genetic loci, and highlights the shared genetic basis with psychiatric traits and brain structure.
Sebastian Markett focused on white matter tract integrity as a potential intermediate phenotype for depression. Analyzing data from the UK Biobank, the study found that depressive symptoms, genetic predisposition for depression, and adverse life events are linked to reduced white matter integrity.
Taken together, these presentations underscore the complex genetic and neurobiological underpinnings of mental disorders and highlight potential intermediate phenotypes through which genetic and environmental risks might affect mental health.
            \begin{description}    
            
                \item [ Konzok J.] Genetics of the Externalizing and Internalizing Dimension: Exploring Genetically Predicted Risk Factors \textcolor{mygray}{ | 09:00}    
                
                \item [ Penate J.] Polygenic prediction of learned fear responses is mediated by functional connectivity in the human extinction network \textcolor{mygray}{ | 09:20}    
                
                \item [ Jawinski P.] Genetics of EEG oscillations reveal novel biological insights into the links between brain structure, brain function, and behavior \textcolor{mygray}{ | 09:40}    
                
                \item [ Markett S.] White Matter Tract Integrity: An Intermediate Phenotype for Depression? \textcolor{mygray}{ | 10:00}    
                
            \end{description} 
            \end{symposium}
            