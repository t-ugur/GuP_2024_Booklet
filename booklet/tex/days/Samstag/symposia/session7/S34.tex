
            \begin{symposium}
            {S34 - Up-regulating and down-regulating memory functions by influencing sleep}
            {Session chair(s): Nora Roüast, Thomas Schreiner}
            {Samstag 09:00 - 10:30 | Ort TODO}
            {Welche Uni TODO}
            Autor Nora Roüast, Co-Autor: Thomas Schreiner

It is well established that sleep is crucial for memory consolidation. The memory function of sleep relies on a delicate interplay of physiological mechanisms facilitating memory reprocessing during sleep. Yet, sleep is not occurring in a vacuum and many factors influence sleep-related neural processes with significant down-stream effects on cognitive functioning. It is therefore crucial to investigate how the sleeping brain codes and reprocesses information, how different factors influence these mechanisms and how this might affect memory functioning.
Within this symposium, we will initially focus on consolidation-related neural processes obtained by intracranial and surface electrophysiology and how experimental interventions such as targeted memory reactivation (TMR) can alter sleep-physiological markers and memory consolidation. We then broaden the perspective by considering how sleep-related consolidation processes and their manipulation can be harnessed to benefit cognitive functioning in both trauma treatment and everyday life.
Firstly, Michael Hahn presents intracranial sleep data on neural population coding efficiency as a mechanism for sleep-related memory consolidation. Thomas Schreiner then examines the relevance of sleep oscillations for TMR triggered memory reactivation using intracranial recordings. Nora Roüast demonstrates that random auditory stimulation can disturb deep sleep physiological markers and declarative memory. Anja Schaich discusses potential benefits of olfactory TMR in sleep on processing traumatic memories and post-traumatic stress disorder treatment. Daniela Ramirez Butavand gives an outlook on how exercise before sleep affects declarative memory performance.
Overall, across multiple methodologies, we demonstrate how different interventions can alter sleep processes and thus memory, whilst proposing potential uses of such approaches.
            \begin{description}    
            
                \item [Michael A. Hahn (TODO: refactor name)] TODO: add title \textcolor{mygray}{ | 09:00}    
                
                \item [Thomas Schreiner  (TODO: refactor name)] TODO: add title \textcolor{mygray}{ | 09:15}    
                
                \item [Nora M. Roüast (TODO: refactor name)] TODO: add title \textcolor{mygray}{ | 09:30}    
                
                \item [Anja Schaich (TODO: refactor name)] TODO: add title \textcolor{mygray}{ | 09:45}    
                
                \item [Daniela Ramirez Butavand (TODO: refactor name)] TODO: add title \textcolor{mygray}{ | 10:00}    
                
            \end{description} 
            \end{symposium}
            