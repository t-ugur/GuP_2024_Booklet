
            \begin{symposium}
            {S32 - Revisiting the relationship between autonomic reactivity and affective and threat-related processes }
            {Session chair(s): Carlos Ventura-Bort }
            {Samstag 09:00 - 10:30 | Ort TODO}
            {Welche Uni TODO}
            Autoren Hedwig Eisenbarth, Maren Klingelhoefer-Jens, Alina Koppold, Carlos Ventura-Bort, 

It is widely acknowledged that feelings of excitement or threat often coincide with physiological changes. However, the complexities surrounding affect-related physiological responses, including their connection to emotional experiences, temporal consistency, and relationship to learning processes, remain topics of ongoing debate. Showcasing a wide range of multivariate methodologies, including machine learning and representational similarity analysis on autonomic (SCR, HR, startle) and BOLD fMRI data, this symposium will provide innovative insights into the dynamics of physiological reactions within affect-inducing contexts.
First, Hedwig Eisenbarth (Victoria University of Wellington) will present data about the contribution of SCR and HR for determining emotional states in both natural and controlled settings. Next, Alina Koppold (University Medical Center Hamburg-Eppendorf) will explore the relationship between valence, arousal, and SCR and startle blink responses, to clarify whether events eliciting similar affective experiences produce comparable physiological reactions. While the autonomic reactivity pattern elicited by established paradigms such as fear conditioning is well-documented, the question remains as to whether these patterns exhibit temporal stability. To address this, Maren Klingelhoefer-Jens (University Medical Center Hamburg – Eppendorf) will present findings on the temporal robustness of SCR and BOLD fMRI evoked by a fear conditioning paradigm. Lastly, Carlos Ventura-Bort (University of Potsdam) will explore the relationship between autonomic reactivity and learning processes, investigating the correspondence between SCR, startle responses, and measures of associative learning change and uncertainty across a series of fear conditioning studies.
            \begin{description}    
            
                \item [Hedwig Eisenbarth (TODO: refactor name)] TODO: add title \textcolor{mygray}{ | 09:00}    
                
                \item [Alina Koppold (TODO: refactor name)] TODO: add title \textcolor{mygray}{ | 09:20}    
                
                \item [Maren Klingelhoefer-Jens (TODO: refactor name)] TODO: add title \textcolor{mygray}{ | 09:40}    
                
                \item [Carlos Ventura-Bort (TODO: refactor name)] TODO: add title \textcolor{mygray}{ | 10:00}    
                
            \end{description} 
            \end{symposium}
            