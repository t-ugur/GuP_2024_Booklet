
\begin{keynote}
    {Body-brain interactions in the control of motivation}
    {Nils B. Kroemer}
    {Saturday 12:00 - 13:00 | ORT}
    {Universität Bonn, Universität Tübingen}

    To ensure survival, optimal reward-seeking requires adaptation to internal and external states, and it is thought that our actions operate on a deeply engrained metabolic budget. Although goal-directed behavior has often been linked to prefrontal circuits, emerging evidence suggests a pivotal role of ascending signals from the body in tuning reward-related behavior according to bodily demands. In this talk, I will review the growing support for bodily signals as key modulators of instrumental behavior and the neural pathways subserving adaptation. First, I will summarize the motivational effects of interventions targeting ascending bodily signals, such as non-invasive transcutaneous vagus nerve stimulation (tVNS). Second, I will discuss the potential mechanistic role of bodily signals, such as gastric myoelectric frequency that regulates the speed of the digestive tract, in the control of motivation. Third, I will evaluate the implications of a focus on body-brain interactions for an improved understanding of the etiology and treatment of frequent mental disorders using major depressive disorder as an example. Fourth, I will highlight remaining challenges and open questions to unlock the potential of novel techniques to effectively modulate goal-directed behavior via the body. Taken together, conceptualizing bodily signals transmitted via vagal afferent as catalysts for goal-directed actions opens new avenues for theory-driven translational work that may help contextualize key motivational symptoms as a result of aberrant body-brain interactions.

    \vspace*{0.5cm}

    \begin{figure}[H]
        \raggedleft
        \includegraphics[width=0.24\textwidth]{tex/images/keynote_speaker/kroemer_cropped.png}
    \end{figure}

\end{keynote}
