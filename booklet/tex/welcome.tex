Liebe Kolleginnen und Kollegen,

wir freuen uns, Sie vom 8.-10. Juni 2023 zur 48. Jahrestagung „Psychologie und Gehirn“ an der Eberhard Karls Universität Tübingen willkommen zu heißen! Die gemeinsam mit der Fachgruppe Biologische Psychologie und Neuropsychologie der Deutschen Gesellschaft für Psychologie (DGPs) und der Deutschen Gesellschaft für Psychophysiologie und ihre Anwendung (DGPA) organisierte Tagung bietet ein breites Spektrum von den neurobiologischen Grundlagen des Verhaltens bis zur anwendungsorientierten psychologischen Forschung. Das diesjährige Tagungsprogramm bietet neben 39 wissenschaftlichen Symposien und zwei Postersessions mit über 300 Posterbeiträgen drei hochkarätige Hauptvorträge mit Prof. Dr. Carmen Sandi, Prof. Dr. Nora Newcombe und Prof. Dr. Onur Güntürkün.

Wir freuen uns sehr Sie in so großer Zahl in Tübingen begrüßen zu dürfen. Wir sehen das auch als Ausdruck der großen Wertschätzung, die wissenschaftliche Tagungen in Präsenz wie die PuG, insbesondere nach den Corona-Jahren, erfahren. Wir hoffen, dass die Tagung, trotz ihrer Größe, den wissenschaftlichen Diskurs in einer persönlichen Atmosphäre ermöglicht. Konferenzen sind dazu da, die Menschen, die hinter den Theorien und Experimenten unseres Faches stehen, kennenzulernen. Und das möchten wir vor allem den jungen aktiv Forschenden unseres Faches ermöglichen. Wir wünschen Ihnen vielfältige Gelegenheiten zu Austausch und Vernetzung und hoffen, dass Sie dabei nicht zuletzt auch etwas Zeit finden, die schwäbische Atmosphäre in unserem „Städtle“ zu genießen.

Jan Born für das gesamte Organisationsteam

Tübingen im Juni 2023

\vspace*{-2cm}

\begin{figure}
	\raggedleft
	\includegraphics[width=0.35\textwidth]{tex/images/jan.png}
\end{figure}


\newpage


\vspace*{2.2cm}

Dear colleagues,

We are pleased to welcome you to the 48th Annual Meeting "Psychologie und Gehirn" at Eberhard Karls University Tübingen from June 8-10, 2023! Organized jointly with the Division of Biological Psychology and Neuropsychology of the German Psychological Society (DGPs) and the German Society for Psychophysiology and its Application (DGPA), the conference will offer a broad spectrum from the neurobiological foundations of behavior to application-oriented psychological research. In addition to 39 scientific symposia and two poster sessions with more than 300 poster contributions, this year's conference program offers three top-class keynote lectures with Prof. Dr. Carmen Sandi, Prof. Dr. Nora Newcombe and Prof. Dr. Onur Güntürkün.

We are very pleased to welcome you in such large numbers to Tübingen. We also see this as an expression of the great esteem in which scientific conferences in presence such as the PuG are held, especially after the Corona years. We hope that the conference, despite its size, will allow scientific discourse in a personal atmosphere. Conferences are about getting to know the people behind the theories and experiments in our field. And we would like to make this possible especially for the young active researchers in our field. We wish you many opportunities for exchange and networking and hope that you will find some time to enjoy the Swabian atmosphere in our „Städtle".

Jan Born for the entire organization team

Tübingen in June 2023

\vspace*{2cm}

\begin{center}
	\includegraphics[width=1\textwidth]{tex/images/tuebingen.png}
\end{center}

\newpage